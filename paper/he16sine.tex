\documentclass[]{article}
\usepackage{CJKutf8}
\usepackage{indentfirst}
\setlength{\parindent}{2em}
\renewcommand{\contentsname}{目录}
\renewcommand{\listfigurename}{插图目录}
\renewcommand{\listtablename}{表格目录}
\renewcommand{\refname}{参考文献}
\renewcommand{\abstractname}{摘要}
\renewcommand{\indexname}{索引}
\renewcommand{\tablename}{表}
\renewcommand{\figurename}{图}
%opening
\date{2011年3月}
\title{基于神经网络的曲线拟合}
\author{何宜晖j\\西安交通大学\\heyihui@stu.xjtu.edu.cn}

\begin{document}
\begin{CJK}{UTF8}{gkai}
%gkai gbsn
\maketitle

\begin{abstract}
人工神经网络是近年来发展起来的模拟人脑生物过程的人工智能技术,具有自学习、自组织、自适应和很强的非线性映射能力。在人工神经网络的实际应用中,常采用BP神经网络或它的变化形式,BP神经网络是一种多层神经网络,因采用BP算法而得名,主要应用于模式识别和分类、函数逼近、数据压缩等领域。  BP网络是一种多层前馈神经网络,由输入层、隐层和输出层组成。层与层之间采用全互连方式,同一层之间不存在相互连接,隐层可以有一个或多个。BP算法的学习过程由前向计算过程和误差反向传播过程组成,在前向计算过程中,输入信息从输入层经隐层逐层计算,并传向输出层,每层神经元的状态只影响下一层神经元的状态。如输出层不能得到期望的输出,则转入误差反向传播过程,误差信号沿原来的连接通路返回,通过修改各层的神经元的权值,使得网络系统误差最小,最终实现网络的实际输出与各自所对应的期望输出逼近。本文利用BP网络修改权值对y=sin(x)曲线实现拟合,并比较了各种实现细节。
\end{abstract}

\section{简介}
要对BP网络进行训练,必须准备训练样本。对样本数据的获取,可以通过用元素列表直接输入、创建数据文件,从数据文件中读取等方式,具体采用哪种方法,取决于数据的多少,数据文件的格式等。  本文采用直接输入100个样本数据的方式,同时采用归一化处理,可以加快网络的训练速度。将输入x和输出y都变为-1到1之间的数据,归一化后的训练样本如下图:

根据系统输入输出序列,确定网络输入层节点数为1,隐含层节点数H为20,输出层节点数为1。初始化输入层、隐含层和输出层神经元之间的连接权值ijw
v,初始化隐含层阈值0t,输出层阈值1t,给定学习速率0a,1a和u,给定算法迭代次数inum和最大可接受误差error,同时给定神经元激励函数sigmoid。
根据输入变量x,输入层和隐含层间连接权值ijw和隐含层阈值0t,计算隐含层输出P:
其中n为输入层节点数,本实验中取1;l为隐含层节点数,本实验中取20;f为隐含层激励函数,该函数可设置为多种形式,本实验中所选函数为:

\section{结论}
通过上面四个图可以看出,随着迭代次数的增加,预测误差是逐渐减小的,在开始的迭代中,误差减小较快,最后经过规定次数的迭代,基本上能将函数y=sin(x)拟合出来。虽然BP神经网络具有较高的拟合能力,但是预测结果仍然存在一定的误差,基本的BP神经网络对于一些复杂系统的预测能力会比较差,其拟合能力存在局限性。
\end{CJK}
\end{document}
